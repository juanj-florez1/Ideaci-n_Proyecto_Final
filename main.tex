\documentclass{article}
\usepackage[utf8]{inputenc}
\usepackage[spanish]{babel}
\usepackage{listings}
\usepackage{graphicx}
\graphicspath{ {images/} }
\usepackage{cite}

\begin{document}

\begin{titlepage}
    \begin{center}
        \vspace*{1cm}
            
        \Huge
        \textbf{Ideación proyecto final }
            
        \vspace{0.5cm}
        \LARGE
        Ideas para el videojuego
            
        \vspace{1.5cm}
            
        \textbf{Jhonny Alejandro Ortiz Osorio\newline C.C: 1001015092}
        
        \textbf{Juan José Florez Argáez\newline C.C: 1001765286}  
        \vfill
            
        \vspace{0.8cm}
            
        \Large
        Despartamento de Ingeniería Electrónica y Telecomunicaciones\\
        Universidad de Antioquia\\
        Medellín\\
        Marzo de 2021
            
    \end{center}
\end{titlepage}

\tableofcontents
\newpage
\section{Introducción}\label{intro}
Este trabajo tiene el objetivo de plasmar las ideas de los dos integrantes del equipo para lograr un videojuego concreto y con una buena historia para que sea entretenido y del agrado de los jugadores.

\section{Ideas} \label{contenido}
A continuación veremos las ideas para el videojuego.
\begin{enumerate}
    \item Que el primer escenario sea en una nave espacial.
    \item Que la nave esté destruida.
    \item El jugador o jugadores deben reparar la nave antes de ser capturados por el enemigo.
    \item Cada reparación a la nave proporcionará puntos.
    \item Habrán comodines o puntos extras si logran completar tareas adicionales dentro de la nave.
    \item El primer nivel termina si el jugador o jugadores lograron reparar la nave.
    \item el segundo nivel.
    \item El jugador o jugadores una vez reparada la nave deben huir del enemigo siendo los mismos jugadores los conductores de la nave.
    \item Deben esquivar los objetos que se interpongan en su camino.
    \item También deben acabar con los enemigos que los persiguen.
    \item Por cada enemigo destruido se asignara puntuación.
    \item El nivel dos es ganado si logran llegar a la tierra a salvo.
    \item Inicia el tercer nivel
    \item Cuando llegan a la tierra deben derrotar el jefe de la raza alienígena Centaurians.
    \item El juego es ganado cuando el líder de los centaurians es vencido.
    
\end{enumerate}


\subsection{}
Vamos a citar por ejemplo un artículo de \textbf{Albert Einstein} \cite{einstein}.
También es posible citar libros \cite{dirac} o documentos en línea \cite{knuthwebsite}.\\\\
Revisar en la última sección el formato de las referencias en IEEE.

\subsection{Incluir código en el documento}
%
A continuación, se presenta el código \ref{codigo_ejemplo}, que nos permite incluir en el informe partes de programa que requieran una explicación adicional.
\begin{lstlisting}[language=C++, label=codigo_ejemplo]
// Programa desarrollado, compilado y ejecutado en https://www.onlinegdb.com
#include <iostream>

/*
 * Esto es un comentario de varias lineas
 */

// Comentario de una sola linea

#define N 10

using namespace std;

int main()
{
    
    for( int i = 0 ; i < N ; i++ ){
        
        if( !(i % 2) )
            cout << " El valor de i es -> " << i << endl;
    }
    
    return 0;
}

//Resultado programa

/*
El valor de i es -> 0
El valor de i es -> 2
El valor de i es -> 4
El valor de i es -> 6
El valor de i es -> 8
*/
\end{lstlisting}
En la sección \ref{imagenes}, se presentará como añadir ilustraciones al texto.

\section{Inclusión de imágenes} \label{imagenes}

En la Figura (\ref{fig:cpplogo}), se presenta el logo de C++ contenido en la carpeta images.

\begin{figure}[h]
\includegraphics[width=4cm]{cpplogo.png}
\centering
\caption{Logo de C++}
\label{fig:cpplogo}
\end{figure}

Las secciones (\ref{intro}), (\ref{contenido}) y (\ref{imagenes}) dependen del estilo del documento.

\bibliographystyle{IEEEtran}
\bibliography{references}

\end{document}
